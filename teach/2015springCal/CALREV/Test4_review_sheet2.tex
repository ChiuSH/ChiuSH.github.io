\documentclass[12pt]{article}
\usepackage[lmargin=.75in,rmargin=.75in,tmargin=.5in,bmargin=1in]{geometry}
\usepackage{amsmath}
\usepackage{amssymb}
\usepackage{amsfonts}
\usepackage{mathrsfs}
\usepackage{theorem}

\usepackage{graphicx}
\usepackage{caption}
\usepackage{subcaption}
\graphicspath{ {images/} }
\usepackage{float}
%\usepackage{psbox}
\usepackage{epstopdf}
%\usepackage{ps2pdf}

%\epstopdfsetup{update}
%\DeclareGraphicsExtensions{.ps}
%\epstopdfDeclareGraphicsRule{.ps}{pdf}{.pdf}{ps2pdf -dEPSCrop -dNOSAFER #1 \OutputFile}

%\usepackage{auto-pst-pdf}
%\usepackage{ucs}
%\usepackage[utf8x]{inputenc}

\usepackage{relsize}

\newtheorem{theorem}{\bf Theorem}[section]
\newtheorem{lemma}[theorem]{Lemma}
\newtheorem{proposition}[theorem]{Proposition}
\newtheorem{corollary}[theorem]{Corollary}
\theorembodyfont{\rmfamily}
\newtheorem{definition}[theorem]{Definition}
\newtheorem{example}[theorem]{Example}
\newtheorem{conjecture}[theorem]{Conjecture}
\newtheorem{remark}[theorem]{Remark}


\newenvironment{proof}{{\bf Proof:}}{\hfill$\square$}
%\renewcommand{\theequation}{\thesection.\arabic{equation}}
%\renewcommand{\thesection}{\arabic{section}}
%\renewcommand{\thesubsection}{(\alph{subsection})}


%\renewcommand{\thesection}{\thechapter .\arabic{section}}
%\newtheorem{theorem}{\bf Theorem}[section]
%\newtheorem{lemma}[theorem]{\bf Lemma}
%\newtheorem{proposition}[theorem]{\bf Proposition}
%\newtheorem{example}[theorem]{\bf Example}
%\newtheorem{definition}[theorem]{\bf Definition}
%\newtheorem{remark}[theorem]{\bf Remark}
%\newtheorem{corollary}[theorem]{\bf Corollary}
%\numberwithin{equation}{chapter}

%\renewcommand{\thesection}{\thechapter .\arabic{section}}

\newcommand{\numbering}[1]{\refstepcounter{theorem}\label{#1}{\noindent \bf\ref{#1}}}

\newcommand{\Adb}{\mbox{$\mathbb{A}$}}
\newcommand{\Bdb}{\mbox{$\mathbb{B}$}}
\newcommand{\Cdb}{\mbox{$\mathbb{C}$}}
\newcommand{\Ddb}{\mbox{$\mathbb{D}$}}
\newcommand{\Edb}{\mbox{$\mathbb{E}$}}
\newcommand{\Fdb}{\mbox{$\mathbb{F}$}}
\newcommand{\Gdb}{\mbox{$\mathbb{G}$}}
\newcommand{\Hdb}{\mbox{$\mathbb{H}$}}
\newcommand{\Idb}{\mbox{$\mathbb{I}$}}
\newcommand{\Jdb}{\mbox{$\mathbb{J}$}}
\newcommand{\Kdb}{\mbox{$\mathbb{K}$}}
\newcommand{\Ldb}{\mbox{$\mathbb{L}$}}
\newcommand{\Mdb}{\mbox{$\mathbb{M}$}}
\newcommand{\Ndb}{\mbox{$\mathbb{N}$}}
\newcommand{\Odb}{\mbox{$\mathbb{O}$}}
\newcommand{\Pdb}{\mbox{$\mathbb{P}$}}
\newcommand{\Qdb}{\mbox{$\mathbb{Q}$}}
\newcommand{\Rdb}{\mbox{$\mathbb{R}$}}
\newcommand{\Sdb}{\mbox{$\mathbb{S}$}}
\newcommand{\Tdb}{\mbox{$\mathbb{T}$}}
\newcommand{\Udb}{\mbox{$\mathbb{U}$}}
\newcommand{\Vdb}{\mbox{$\mathbb{V}$}}
\newcommand{\Wdb}{\mbox{$\mathbb{W}$}}
\newcommand{\Xdb}{\mbox{$\mathbb{X}$}}
\newcommand{\Ydb}{\mbox{$\mathbb{Y}$}}
\newcommand{\Zdb}{\mbox{$\mathbb{Z}$}}

\newcommand{\A}{\mbox{$\mathscr{A}$}}
%\newcommand{\A}{\mbox{${\mathcal A}$}}
\newcommand{\B}{\mbox{${\mathcal B}$}}
\newcommand{\C}{\mbox{${\mathcal C}$}}
\newcommand{\D}{\mbox{${\mathcal D}$}}
\newcommand{\E}{\mbox{${\mathcal E}$}}
\newcommand{\F}{\mbox{${\mathcal F}$}}
\newcommand{\G}{\mbox{${\mathcal G}$}}
\renewcommand{\H}{\mbox{${\mathcal H}$}}
\newcommand{\I}{\mbox{${\mathcal I}$}}
\newcommand{\J}{\mbox{${\mathcal J}$}}
\newcommand{\K}{\mbox{${\mathcal K}$}}
\newcommand{\Ll}{\mbox{${\mathcal L}$}}
\newcommand{\Mc}{\mbox{${\mathcal M}$}}
\newcommand{\N}{\mbox{${\mathcal N}$}}
\renewcommand{\O}{\mbox{${\mathcal O}$}}
\renewcommand{\P}{\mbox{${\mathcal P}$}}
\newcommand{\Q}{\mbox{${\mathcal Q}$}}
\newcommand{\R}{\mbox{${\mathcal R}$}}
\renewcommand{\S}{\mbox{${\mathcal S}$}}
\newcommand{\T}{\mbox{${\mathcal T}$}}
\newcommand{\U}{\mbox{${\mathcal U}$}}
\newcommand{\V}{\mbox{${\mathcal V}$}}
\newcommand{\W}{\mbox{${\mathcal W}$}}
\newcommand{\X}{\mbox{${\mathcal X}$}}
\newcommand{\Y}{\mbox{${\mathcal Y}$}}
\newcommand{\Z}{\mbox{${\mathcal Z}$}}

\DeclareMathOperator{\arcsec}{arcsec}
\DeclareMathOperator{\arccot}{arccot}
\DeclareMathOperator{\arccsc}{arccsc}

\newcommand\numberthis{\addtocounter{equation}{1}\tag{\theequation}}

\begin{document}

%\title{\Large Math 1431, Calculus I Test 4 Review, Spring 2015.}
%\author{ Shang-Huan Chiu }

%\maketitle





\section{Riemann Sum}
Given a continuous function $f$ and a partition $P=\{x_0=a, x_1, x_2, \cdots, x_{n-1}, x_n=b\}$ on $[a,b]$.
Then we can estimate $\int_{a}^{b} f(x)\, dx$ by Riemann Sum:
$$\sum [(\text{ length of the subinterval})\times (\text{ value of $f$ on this subinterval})]$$
\begin{itemize}
\item[(1)] Upper sum ($U_f$)
$$ U_f=\sum [(\text{ length of the subinterval})\times (\text{ maximum value of $f$ on this subinterval})]$$

\item[(2)] Lower sum ($L_f$)
$$ L_f=\sum [(\text{ length of the subinterval})\times (\text{ minimum value of $f$ on this subinterval})]$$

\item[(3)] $U_f \geq L_f$

\item[(2)] Specific points:(left endpoint, right endpoint, midpoint) 
$$ \text{ Sum }=\sum [(\text{ length of the subinterval})\times (\text{ specific point value of $f$ on this subinterval})]$$

\end{itemize}
\section{Basic Integration Properties}
Assume $f,g$ are continuous on $[a,b]$ and $\alpha, \beta$ are constants.
\begin{itemize}
\item[(1)] If $a<c<b$, then
$$\int_{a}^{c} f(x)\, dx+\int_{c}^{b} f(x)\, dx=\int_{a}^{b} f(x)\, dx.$$

\item[(2)] The integration value will change of sign if we integrate in the different directions:
$$\int_{b}^{a} f(x)\, dx=-\int_{a}^{b} f(x)\, dx.$$

\item[(3)] The integral from any number to itself is defined to be zero:
$$\int_{c}^{c} f(x)\, dx=0.$$

\item[(4)] Linearity of integration:

$$\int_{a}^{b} \alpha f(x) +\beta g(x)\, dx=\alpha \int_{a}^{b} f(x)\, dx + \beta \int_{a}^{b} g(x)\, dx.$$



\end{itemize}

\clearpage

\section{ Fundamental Theorem of Calculus}

\subsection{ First Fundamental Theorem of Calculus}
\begin{theorem}
Let $f$ be continuous on $[a,b]$. The function $F$ defined on $[a,b]$ by 
$$F(x)=\int_{a}^{x} f(t)\, dt$$
is continuous on $[a,b]$, differentiable on $(a,b)$, and has derivative
$F'(x)=f(x)$ for all $x$ in $(a,b)$.
\end{theorem}

Assume a function $f$ is defined as above in the theorem and function $u(x), v(x)$ are differentiable, we have
\begin{itemize}
\item[(1)] If $F(x)=\int_{a}^{x} f(t)\, dt$, then $$F'(x)=f(x).$$

\item[(2)] If $F(x)=\int_{x}^{a} f(t)\, dt$, so $F(x)=-\int_{a}^{x} f(t)\, dt$, then
$$F'(x)=-f(x).$$

\item[(3)] If $F(x)=\int_{a}^{u(x)} f(t)\, dt$, then 
$$F'(x)=u'(x)\cdot f(u(x)).$$

\item[(3)] If $F(x)=\int_{v(x)}^{u(x)} f(t)\, dt$, there is a constant $c$ such that
$$F(x)=\int_{c}^{u(x)} f(t)\, dt- \int_{c}^{v(x)} f(t)\, dt$$
then we have 
$$F'(x)=u'(x)\cdot f(u(x))- v'(x)\cdot f(v(x)).$$
\end{itemize}

\subsection{ Second Fundamental Theorem of Calculus}
\begin{definition}
Let $f$ be continuous on $[a,b]$. A function is called an antiderivative for $f$ on $[a,b]$ if 
$$F \text{ is continuous on } [a,b] \text{ and } F'(x)=f(x) \text{ for all } x \in (a,b).$$
\end{definition}

\begin{theorem}
Let $f$ be continuous on $[a,b]$. If $F$ is any antiderivative for $f$ on $[a,b]$, then
$$\int_{a}^{b} f(t)\, dt = F(b)-F(a).$$
\end{theorem}

\section{Differential}
Given $f(a+h)$. we can estimate $f(a+h) -f(a)$ by differential $df$:

$$f(a+h)-f(a) \approx df= f'(a)\cdot h,$$ then
$$f(a+h) \approx f(a)+ f'(a)\cdot h.$$





























\end{document}