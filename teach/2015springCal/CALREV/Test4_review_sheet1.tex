\documentclass[12pt]{article}
\usepackage[lmargin=.2in,rmargin=.70in,tmargin=.5in,bmargin=1in]{geometry}
\usepackage{amsmath}
\usepackage{amssymb}
\usepackage{amsfonts}
\usepackage{mathrsfs}
\usepackage{theorem}

\usepackage{graphicx}
\usepackage{caption}
\usepackage{subcaption}
\graphicspath{ {images/} }
\usepackage{float}
%\usepackage{psbox}
\usepackage{epstopdf}
%\usepackage{ps2pdf}

%\epstopdfsetup{update}
%\DeclareGraphicsExtensions{.ps}
%\epstopdfDeclareGraphicsRule{.ps}{pdf}{.pdf}{ps2pdf -dEPSCrop -dNOSAFER #1 \OutputFile}

%\usepackage{auto-pst-pdf}
%\usepackage{ucs}
%\usepackage[utf8x]{inputenc}

\usepackage{relsize}

\newtheorem{theorem}{\bf Theorem}[section]
\newtheorem{lemma}[theorem]{Lemma}
\newtheorem{proposition}[theorem]{Proposition}
\newtheorem{corollary}[theorem]{Corollary}
\theorembodyfont{\rmfamily}
\newtheorem{definition}[theorem]{Definition}
\newtheorem{example}[theorem]{Example}
\newtheorem{conjecture}[theorem]{Conjecture}
\newtheorem{remark}[theorem]{Remark}


\newenvironment{proof}{{\bf Proof:}}{\hfill$\square$}
%\renewcommand{\theequation}{\thesection.\arabic{equation}}
%\renewcommand{\thesection}{\arabic{section}}
%\renewcommand{\thesubsection}{(\alph{subsection})}


%\renewcommand{\thesection}{\thechapter .\arabic{section}}
%\newtheorem{theorem}{\bf Theorem}[section]
%\newtheorem{lemma}[theorem]{\bf Lemma}
%\newtheorem{proposition}[theorem]{\bf Proposition}
%\newtheorem{example}[theorem]{\bf Example}
%\newtheorem{definition}[theorem]{\bf Definition}
%\newtheorem{remark}[theorem]{\bf Remark}
%\newtheorem{corollary}[theorem]{\bf Corollary}
%\numberwithin{equation}{chapter}

%\renewcommand{\thesection}{\thechapter .\arabic{section}}

\newcommand{\numbering}[1]{\refstepcounter{theorem}\label{#1}{\noindent \bf\ref{#1}}}

\newcommand{\Adb}{\mbox{$\mathbb{A}$}}
\newcommand{\Bdb}{\mbox{$\mathbb{B}$}}
\newcommand{\Cdb}{\mbox{$\mathbb{C}$}}
\newcommand{\Ddb}{\mbox{$\mathbb{D}$}}
\newcommand{\Edb}{\mbox{$\mathbb{E}$}}
\newcommand{\Fdb}{\mbox{$\mathbb{F}$}}
\newcommand{\Gdb}{\mbox{$\mathbb{G}$}}
\newcommand{\Hdb}{\mbox{$\mathbb{H}$}}
\newcommand{\Idb}{\mbox{$\mathbb{I}$}}
\newcommand{\Jdb}{\mbox{$\mathbb{J}$}}
\newcommand{\Kdb}{\mbox{$\mathbb{K}$}}
\newcommand{\Ldb}{\mbox{$\mathbb{L}$}}
\newcommand{\Mdb}{\mbox{$\mathbb{M}$}}
\newcommand{\Ndb}{\mbox{$\mathbb{N}$}}
\newcommand{\Odb}{\mbox{$\mathbb{O}$}}
\newcommand{\Pdb}{\mbox{$\mathbb{P}$}}
\newcommand{\Qdb}{\mbox{$\mathbb{Q}$}}
\newcommand{\Rdb}{\mbox{$\mathbb{R}$}}
\newcommand{\Sdb}{\mbox{$\mathbb{S}$}}
\newcommand{\Tdb}{\mbox{$\mathbb{T}$}}
\newcommand{\Udb}{\mbox{$\mathbb{U}$}}
\newcommand{\Vdb}{\mbox{$\mathbb{V}$}}
\newcommand{\Wdb}{\mbox{$\mathbb{W}$}}
\newcommand{\Xdb}{\mbox{$\mathbb{X}$}}
\newcommand{\Ydb}{\mbox{$\mathbb{Y}$}}
\newcommand{\Zdb}{\mbox{$\mathbb{Z}$}}

\newcommand{\A}{\mbox{$\mathscr{A}$}}
%\newcommand{\A}{\mbox{${\mathcal A}$}}
\newcommand{\B}{\mbox{${\mathcal B}$}}
\newcommand{\C}{\mbox{${\mathcal C}$}}
\newcommand{\D}{\mbox{${\mathcal D}$}}
\newcommand{\E}{\mbox{${\mathcal E}$}}
\newcommand{\F}{\mbox{${\mathcal F}$}}
\newcommand{\G}{\mbox{${\mathcal G}$}}
\renewcommand{\H}{\mbox{${\mathcal H}$}}
\newcommand{\I}{\mbox{${\mathcal I}$}}
\newcommand{\J}{\mbox{${\mathcal J}$}}
\newcommand{\K}{\mbox{${\mathcal K}$}}
\newcommand{\Ll}{\mbox{${\mathcal L}$}}
\newcommand{\Mc}{\mbox{${\mathcal M}$}}
\newcommand{\N}{\mbox{${\mathcal N}$}}
\renewcommand{\O}{\mbox{${\mathcal O}$}}
\renewcommand{\P}{\mbox{${\mathcal P}$}}
\newcommand{\Q}{\mbox{${\mathcal Q}$}}
\newcommand{\R}{\mbox{${\mathcal R}$}}
\renewcommand{\S}{\mbox{${\mathcal S}$}}
\newcommand{\T}{\mbox{${\mathcal T}$}}
\newcommand{\U}{\mbox{${\mathcal U}$}}
\newcommand{\V}{\mbox{${\mathcal V}$}}
\newcommand{\W}{\mbox{${\mathcal W}$}}
\newcommand{\X}{\mbox{${\mathcal X}$}}
\newcommand{\Y}{\mbox{${\mathcal Y}$}}
\newcommand{\Z}{\mbox{${\mathcal Z}$}}

\DeclareMathOperator{\arcsec}{arcsec}
\DeclareMathOperator{\arccot}{arccot}
\DeclareMathOperator{\arccsc}{arccsc}

\newcommand\numberthis{\addtocounter{equation}{1}\tag{\theequation}}

\begin{document}

\title{\Large Math 1431, Calculus I Test 4 Review, Spring 2015.}
\author{ Shang-Huan Chiu }

\maketitle



\section{Differentiation and Integration Tables}

Assume $n$, $a$, and $C$ are constants. $(\dag) 1-x^2 >0$. $(\ddag) x^2-1>0$
\begin{figure}[htb]\large
 \begin{minipage}[b]{0.5\textwidth} 
  \renewcommand\arraystretch{1.5}%表格的上下高度
%\begin{table}[h]\large%字體大小  
\centering 
\begin{tabular}{|c|c|}
\hline
Function                                 & Derivative of given function \\ \hline
$x^n$, $n \neq 0$ 	& $n\cdot x^{n-1}$                   \\ \hline
$\ln (x)$                                       & $\frac{1}{x}$                    \\ \hline
$e^x$                                       & $e^x$                    \\ \hline
$\sin(x)$                                       & $\cos(x)$                    \\ \hline
$\cos(x)$                                       & $-\sin(x)$                   \\ \hline
$\tan(x)$                                       & $\sec^2(x)$                  \\ \hline
$\sec(x)$                                       & $\sec(x) \tan(x)$            \\ \hline
$\cot(x)$                                       & $-\csc^2(x)$                 \\ \hline
$\csc(x)$                                       & $-\csc(x) \cot(x)$           \\ \hline
$\sinh(x)$                                       & $\cosh(x)$                    \\ \hline
$\cosh(x)$                                       & $\sinh(x)$                   \\ \hline
 $\arcsin(x)$                                      & $\frac{1}{\sqrt{1-x^2}}$                  \\ \hline
 $\arctan(x)$                                                      &   $\frac{1}{1+x^2}$     \\ \hline
 $\arcsec(x)$                                               & $\frac{1}{|x| \sqrt{x^2-1}}$                             \\ \hline
 $a^x$,$a>0$                                              &  $ a^x \ln (a)$                            \\ \hline


\end{tabular}
\caption{Differentiation Table} 
\end{minipage}% 
  \begin{minipage}[b]{0.5\textwidth} 
    \centering 
\renewcommand\arraystretch{1.5}
\begin{tabular}{|c|c|}
\hline
Function                                  & Integral of given function \\ \hline
$x^n$    					 	& \begin{minipage}{7.88cm} \vspace{0.2cm}\centering $\mathop{\mathlarger{\int}} x^n\, dx=$ 
$
\begin{cases}
\frac{x^{n+1}}{n+1} +C,  &  \text{ if }n \neq -1;\\
\ln|x| +C,                          &  \text{ if }n= -1.
\end{cases}
$\vspace{0.2cm}\end{minipage}
                \\ \hline
$e^{x}$                                       & $\int e^x \,dx = e^x +C$                    \\ \hline
$\cos(x)$                                       & $\int \cos(x) \, dx= \sin(x)+C$                    \\ \hline
$\sin(x)$                                       & $\int \sin(x) \, dx= -\cos(x)+C$                    \\ \hline
$\sec^2(x)$                                       & $\int \sec^2(x) \, dx= \tan(x)+C$                   \\ \hline
$\sec(x) \tan(x)$                                      & $\int \sec(x) \tan(x) \, dx= \sec(x)+C$                  \\ \hline
 $\csc^2(x)$                                  &$\int \csc^2(x) \, dx= -\cot(x)+C$                  \\ \hline
$\csc(x) \cot(x)$                                      &$\int \csc(x) \cot(x) \, dx= -\csc(x)+C$            \\ \hline
$\cosh(x)$                                      &$\int \cosh(x) \, dx= \sinh(x)+C$         \\ \hline
$\sinh(x)$                                       &$\int \sinh(x) \, dx= \cosh(x)+C$        \\ \hline
$\frac{1}{\sqrt{1-x^2}}$$(\dag)$                 &$\int \frac{1}{\sqrt{1-x^2}} \, dx= \arcsin(x)+C$        \\ \hline
$\frac{1}{1+x^2}$                          &$\int \frac{1}{1+x^2} \, dx= \arctan(x)+C$        \\ \hline
$\frac{1}{x \sqrt{x^2-1}}$$(\ddag)$           &$\int \frac{1}{x \sqrt{x^2-1}} \, dx= \arcsec(x)+C$        \\ \hline
$a^x$,$a>0$                                                      &$\int a^x \, dx= \frac{1}{\ln(a)} a^x+c$        \\ \hline

   \end{tabular}
\caption{Integration Table} 
  \end{minipage} 
\end{figure}
%\end{table}

\end{document}