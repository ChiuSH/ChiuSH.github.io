\documentclass[12pt]{article}
\usepackage[lmargin=.75in,rmargin=.75in,tmargin=.5in,bmargin=1in]{geometry}
\usepackage{amsmath}
\usepackage{amssymb}
\usepackage{amsfonts}
\usepackage{mathrsfs}
\usepackage{theorem}

\usepackage{graphicx}
\usepackage{caption}
\usepackage{subcaption}
\graphicspath{ {images/} }
\usepackage{float}
%\usepackage{psbox}
\usepackage{epstopdf}
%\usepackage{ps2pdf}

%\epstopdfsetup{update}
%\DeclareGraphicsExtensions{.ps}
%\epstopdfDeclareGraphicsRule{.ps}{pdf}{.pdf}{ps2pdf -dEPSCrop -dNOSAFER #1 \OutputFile}

%\usepackage{auto-pst-pdf}
%\usepackage{ucs}
%\usepackage[utf8x]{inputenc}


\newtheorem{theorem}{\bf Theorem}[section]
\newtheorem{lemma}[theorem]{Lemma}
\newtheorem{proposition}[theorem]{Proposition}
\newtheorem{corollary}[theorem]{Corollary}
\theorembodyfont{\rmfamily}
\newtheorem{definition}[theorem]{Definition}
\newtheorem{example}[theorem]{Example}
\newtheorem{conjecture}[theorem]{Conjecture}
\newtheorem{remark}[theorem]{Remark}


\newenvironment{proof}{{\bf Proof:}}{\hfill$\square$}
%\renewcommand{\theequation}{\thesection.\arabic{equation}}
%\renewcommand{\thesection}{\arabic{section}}
%\renewcommand{\thesubsection}{(\alph{subsection})}


%\renewcommand{\thesection}{\thechapter .\arabic{section}}
%\newtheorem{theorem}{\bf Theorem}[section]
%\newtheorem{lemma}[theorem]{\bf Lemma}
%\newtheorem{proposition}[theorem]{\bf Proposition}
%\newtheorem{example}[theorem]{\bf Example}
%\newtheorem{definition}[theorem]{\bf Definition}
%\newtheorem{remark}[theorem]{\bf Remark}
%\newtheorem{corollary}[theorem]{\bf Corollary}
%\numberwithin{equation}{chapter}

%\renewcommand{\thesection}{\thechapter .\arabic{section}}

\newcommand{\numbering}[1]{\refstepcounter{theorem}\label{#1}{\noindent \bf\ref{#1}}}

\newcommand{\Adb}{\mbox{$\mathbb{A}$}}
\newcommand{\Bdb}{\mbox{$\mathbb{B}$}}
\newcommand{\Cdb}{\mbox{$\mathbb{C}$}}
\newcommand{\Ddb}{\mbox{$\mathbb{D}$}}
\newcommand{\Edb}{\mbox{$\mathbb{E}$}}
\newcommand{\Fdb}{\mbox{$\mathbb{F}$}}
\newcommand{\Gdb}{\mbox{$\mathbb{G}$}}
\newcommand{\Hdb}{\mbox{$\mathbb{H}$}}
\newcommand{\Idb}{\mbox{$\mathbb{I}$}}
\newcommand{\Jdb}{\mbox{$\mathbb{J}$}}
\newcommand{\Kdb}{\mbox{$\mathbb{K}$}}
\newcommand{\Ldb}{\mbox{$\mathbb{L}$}}
\newcommand{\Mdb}{\mbox{$\mathbb{M}$}}
\newcommand{\Ndb}{\mbox{$\mathbb{N}$}}
\newcommand{\Odb}{\mbox{$\mathbb{O}$}}
\newcommand{\Pdb}{\mbox{$\mathbb{P}$}}
\newcommand{\Qdb}{\mbox{$\mathbb{Q}$}}
\newcommand{\Rdb}{\mbox{$\mathbb{R}$}}
\newcommand{\Sdb}{\mbox{$\mathbb{S}$}}
\newcommand{\Tdb}{\mbox{$\mathbb{T}$}}
\newcommand{\Udb}{\mbox{$\mathbb{U}$}}
\newcommand{\Vdb}{\mbox{$\mathbb{V}$}}
\newcommand{\Wdb}{\mbox{$\mathbb{W}$}}
\newcommand{\Xdb}{\mbox{$\mathbb{X}$}}
\newcommand{\Ydb}{\mbox{$\mathbb{Y}$}}
\newcommand{\Zdb}{\mbox{$\mathbb{Z}$}}

\newcommand{\A}{\mbox{$\mathscr{A}$}}
%\newcommand{\A}{\mbox{${\mathcal A}$}}
\newcommand{\B}{\mbox{${\mathcal B}$}}
\newcommand{\C}{\mbox{${\mathcal C}$}}
\newcommand{\D}{\mbox{${\mathcal D}$}}
\newcommand{\E}{\mbox{${\mathcal E}$}}
\newcommand{\F}{\mbox{${\mathcal F}$}}
\newcommand{\G}{\mbox{${\mathcal G}$}}
\renewcommand{\H}{\mbox{${\mathcal H}$}}
\newcommand{\I}{\mbox{${\mathcal I}$}}
\newcommand{\J}{\mbox{${\mathcal J}$}}
\newcommand{\K}{\mbox{${\mathcal K}$}}
\newcommand{\Ll}{\mbox{${\mathcal L}$}}
\newcommand{\Mc}{\mbox{${\mathcal M}$}}
\newcommand{\N}{\mbox{${\mathcal N}$}}
\renewcommand{\O}{\mbox{${\mathcal O}$}}
\renewcommand{\P}{\mbox{${\mathcal P}$}}
\newcommand{\Q}{\mbox{${\mathcal Q}$}}
\newcommand{\R}{\mbox{${\mathcal R}$}}
\renewcommand{\S}{\mbox{${\mathcal S}$}}
\newcommand{\T}{\mbox{${\mathcal T}$}}
\newcommand{\U}{\mbox{${\mathcal U}$}}
\newcommand{\V}{\mbox{${\mathcal V}$}}
\newcommand{\W}{\mbox{${\mathcal W}$}}
\newcommand{\X}{\mbox{${\mathcal X}$}}
\newcommand{\Y}{\mbox{${\mathcal Y}$}}
\newcommand{\Z}{\mbox{${\mathcal Z}$}}

\DeclareMathOperator{\arcsec}{arcsec}
\DeclareMathOperator{\arccot}{arccot}
\DeclareMathOperator{\arccsc}{arccsc}

\newcommand\numberthis{\addtocounter{equation}{1}\tag{\theequation}}

\begin{document}

\title{\Large Math 1431, Calculus I Final Review, Spring 2015.}
\author{ Shang-Huan Chiu }

\maketitle



\section{Existence of Limit}
Given a real function $f(x)$. Then $\lim\limits_{x\to a} f(x)$, the limit of $f(x)$ at $x=a$, exists if 
$$ \lim\limits_{x\to a^+} f(x) =\lim\limits_{x\to a^-} f(x),$$ so we have
$$\lim\limits_{x\to a} f(x) = \lim\limits_{x\to a^+} f(x) =\lim\limits_{x\to a^-} f(x).$$

\section{Properties of Limit}
\subsection{Basic Properties}

Given real functions $f(x)$, $g(x)$. 
Assume the limits of $f$, $g$ exist at a point $a$, 
i.e. $\lim\limits_{x\to a} f(x)$, $\lim\limits_{x\to a} g(x)$ exist.
\begin{itemize}
\item $\lim\limits_{x\to a}[ f(x) +g(x)] =\lim\limits_{x\to a} f(x) +\lim\limits_{x\to a} g(x)$.
 
\item $\lim\limits_{x\to a}[ f(x)-g(x)] =\lim\limits_{x\to a} f(x)-\lim\limits_{x\to a} g(x)$.

\item $\lim\limits_{x\to a} [f(x)g(x)] = [\lim\limits_{x\to a} f(x)] \cdot [\lim\limits_{x\to a} g(x)].$

\item If $\lim\limits_{x\to a} g(x) \neq 0$, we have $\lim\limits_{x\to a} \dfrac{f(x)}{g(x)} = \dfrac{\lim\limits_{x\to a} f(x)}{\lim\limits_{x\to a} g(x)}$.

\item If $f$ is a continuous function on $\Rdb$, then we have 
$\lim\limits_{x\to a} f({g(x)})= f({\lim\limits_{x\to a} g(x)}).$

\begin{example}
Let $f(x) = e^x$. Then $\lim\limits_{x\to a} e^{g(x)}= e^{\lim\limits_{x\to a} g(x)}.$
\end{example}
\end{itemize}

\subsection{Special Properity}

Given two polynomials $P(x)$, $Q(x)$. Define $\deg(P)$ be the degree of  polynomial $P(x)$ 
which is the highest degree of its terms. Then we have

\[
\lim_{x \to \infty }\dfrac{P(x)}{Q(x)}=
\begin{cases}
0,             & \text{ if } \deg(P(x)) < \deg(Q(x)),\\
\text{Leading Coefficient}, & \text{ if }\deg(P(x)) = \deg(Q(x)),\\
\infty,       &  \text{ if } \deg(P(x)) > \deg(Q(x)).
\end{cases}
\]
where the leading coefficient is the ratio of the cofficients of the highest degree terms of $P(x)$ and $Q(x)$.


\section{Continuity and Discontinuity of A Real Function}

\begin{definition}
Given a real function $f(x)$. Then $f$ is said to be continuous at a point $a$ if 
for every $\epsilon >0$ there exists a $\delta >0 $ such that 
$$|x-a| < \delta  \text{ implies } |f(x)-f(a)| < \epsilon.$$
\end{definition}

\subsection{Classification}
To classify the continuity or discontinuity of a real function at a point $a$, 
we can check the three values
$$\text{(1)}\lim_{x \to a^+}f(x), \text{ (2)}\lim_{x \to a^-}f(x), \text{ (3)}f(a)$$ 
and we have the following four conditions:
\begin{itemize}
\item If (1), (2), (3) exist and (1) $=$ (2) $=$ (3), then $f$ is {\bf continuous} at $a$.

\item If (1), (2) exist and (1) $=$ (2) $\neq$ (3), then $f$ has a {\bf removable discontinuity} at $a$.

\item If (1), (2) exist and (1) $\neq$ (2), then $f$ has a {\bf jump discontinuity} at $a$.

\item If at least one of the first two values tends to $\infty$ or $-\infty$, 
then $f$ has an {\bf infinite discontinuity} at $a$.

\end{itemize}


\section{The Intermediate Value Theorem}

\begin{theorem}
If $f(x)$ is continuous on the closed interval $[a,b]$ and $N$ is a value between $f(a)$ and $f(b)$,
then there is at least one value $c$ in $(a,b)$ such that
$$f(c)=N.$$
\end{theorem}

\begin{corollary}[Root finding]
If $f(x)$ is continuous on the closed interval $[a,b]$ and $f(a)$, $f(b)$ have different signs, that is
$$f(a) < 0 <f(b), \text{ or } f(b) < 0 <f(a),$$
then there is at least one value $c$ in $(a,b)$ such that
$$f(c)=0.$$
\end{corollary}

\section{The Pinching Theorem}
\begin{theorem}
Suppose $f(x), g(x),$ and $h(x)$ are defined in an open interval containing $x=a$.
If $f(x) \leq g(x) \leq h(x)$  and $\lim\limits_{x\to a} f(x)=\lim\limits_{x\to a} h(x)=L$, then
$$\lim\limits_{x\to a} g(x)=L.$$
\end{theorem}

\subsection{Application of Pinching Theorem}
\begin{itemize}
\item $\lim\limits_{x\to 0} \dfrac{sin(ax)}{ax}=1.$

\item $\lim\limits_{x\to 0} \dfrac{1-\cos(ax)}{ax}=0.$
\end{itemize}
(We also can use L'H$\hat{\text{o}}$pital's rule to find these.)

\section{Sum and Difference Formulas}
$$ \sin(A \pm B)=\sin(A)\cos(B) \pm \cos(A)\sin(B). $$
$$ \cos(A \pm B)=\cos(A)\cos(B) \mp \sin(A)\sin(B).$$

\section{Derivative}

\subsection{Definition of derivative}

A function $f(x)$ is differentiable at $x$ if and only if 
$$\lim\limits_{ h \to 0} \dfrac{f(x+h)-f(x)}{h}$$
exists. We denote
$$f'(x)=\lim\limits_{ h \to 0} \dfrac{f(x+h)-f(x)}{h}$$
and we refer to $f'(x)$ as the derivative of $f$ at $x$.



\subsection{Product Rule}
Assume $f(x),g(x)$ are differentiable. Then
$$\frac{d}{dx}[f(x)\cdot g(x)]=f'(x)\cdot g(x) + f(x)\cdot g'(x).$$

\subsection{Quotient Rule}
Assume $f(x),g(x)$ are differentiable. Then
$$ \frac{d}{dx}\left[\dfrac{f(x)}{g(x)}\right]=\dfrac{g(x)f'(x)-f(x)g(x)'}{[g(x)]^2}.$$
(lo D-hi minus hi D-lo)

\subsection{Chain Rule}
Assume $f(x),g(x)$ are differentiable. Then 
$$\frac{d}{dx}[f(g(x))]=f'(g(x))g'(x) $$

Application of Chain Rule:
Assume $u(x)$ is differentiable. 
\begin{figure}[htb]\large
 %\begin{minipage}[b]{textwidth} 
  \renewcommand\arraystretch{1.5}%表格的上下高度
%\begin{table}[H]\large%字體大小  
\centering 
\begin{tabular}{|c|c|}
\hline
Function                                 & Derivative of given function \\ \hline
$[u(x)]^n$, $n \neq 0$ 	& $n \cdot u'(x) [u(x)]^{n-1}$                   \\ \hline
$\ln u(x)$                                       & $\dfrac{u'(x)}{u(x)}$                    \\ \hline
$e^{u(x)}$                                       & $u'(x) e^{u(x)}$                    \\ \hline
$\sin(u(x))$                                       & $u'(x)\cos(u(x))$                    \\ \hline
$\cos(u(x))$                                       & $-u'(x)\sin(u(x))$                   \\ \hline
$\tan(u(x))$                                       & $u'(x)\sec^2(u(x))$                  \\ \hline
$\sec(u(x))$                                       & $u'(x)\sec(u(x)) \tan(u(x))$            \\ \hline
$\cot(u(x))$                                       & $-u'(x)\csc^2(u(x))$                 \\ \hline
$\csc(u(x))$                                       & $-u'(x)\csc(u(x)) \cot(u(x))$           \\ \hline
$\sinh(u(x))$                                       & $u'(x)\cosh(u(x))$                    \\ \hline
$\cosh(u(x))$                                       & $u'(x)\sinh(u(x))$                   \\ \hline
 $\arcsin(u(x))$                                      & $\dfrac{u'(x)}{\sqrt{1-[u(x)]^2}}$                  \\ \hline
 $\arctan(u(x))$                                                      &   $\dfrac{u'(x)}{1+[u(x)]^2}$     \\ \hline
 $\arcsec(u(x))$                                               & $\dfrac{u'(x)}{|u(x)| \sqrt{[u(x)]^2-1}}$                             \\ \hline
 $a^{u(x)}$,$a>0$                                              &  $ \ln (a) u'(x)a^{u(x)} $                            \\ \hline


\end{tabular}
\caption{Differentiation Table} 
%\end{table}% 
\end{figure}

(Please check the test4 review sheet table1)

\clearpage 
\section{Special Formulas for Areas and Volumes}
\begin{itemize}
\item Area of a {\bf circle} with a radius $r$: $$A=\pi r^2.$$

\item Volume of a {\bf cone} with height $h$ and radius of base $r$: $$V=\frac{1}{3} \pi r^2 h.$$

\item Volume of a {\bf ball} with radius $r$:
$$V=\frac{4}{3} \pi r^3.$$

\item Area of {\bf sphere} with radius $r$:
$$S= 4 \pi r^2.$$

\item Surface area of {\bf cylinder} with height $h$ and radius of base $r$:
$$ S=2\pi r h +2 \pi r^2.$$ 


\end{itemize}

\section{Mean Value Theorem}
\begin{theorem}[Mean Value Theorem]
If $f$ is continuous on a $[a,b]$ and differentiable on $(a,b)$,
then there exists a number $c$ in $(a,b)$ such that
$$ f'(c)=\dfrac{f(b)-f(a)}{b-a}.$$
\end{theorem}

\begin{theorem}[Rolle's Theorem]
Let $f$ be continuous on a $[a,b]$ and differentiable on $(a,b)$.
If $f(a)=f(b)$ then there is at least one number $c$ in $(a,b)$ such that
$$f'(c)=0$$
\end{theorem}








































































\end{document}